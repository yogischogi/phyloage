\section{Installation}

This guide is mainly targeted towards persons who use Linux Mint
or other Linux versions of the Debian family. Some familiarity
with the use of Linux commands is assumed.

Currently there are no binary distributions available for
Windows or the Mac. Users of these operating systems can
use Phyloage as well, but they will experience some
laborious installation work. The best way is to follow the
instructions provided on the
\href{http://golang.org/}{Go} home page.

The following list applies to Linux users only:

\begin{enumerate}
\item Make sure that the Go programming language is installed.
	If not it can be installed by typing\\
	\texttt{sudo apt-get install golang}
\item Read the Go
	\href{http://golang.org/doc/install}{Getting Started}
	guide. Make sure to set your \emph{GOPATH} variable and
	include it in your \emph{PATH} so that Go programs can be
	found.
\item Fetch the Phyloage program with\\
	\texttt{go get github.com/yogischogi/phyloage}
\item Install the program with\\
	 \texttt{go install github.com/yogischogi/phyloage}
\end{enumerate}

You should end up with two newly installed program, Phyloage
and Phylofriend \cite{Phylofriend}. Phylofriend does a lot
of the background calculations for Phyloage.

