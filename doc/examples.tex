\section{Examples}

\subsection{Using YFull results}

\subsubsection*{SNP input tree}

Before you can start you need to create a phylogenetic tree,
that contains SNPs and the IDs of the genetic samples. The
file format is text based and looks like this:

\begin{verbatim}
// This is an example tree.
// Comments begin with //.

CTS4528, S1200
    S11481
        id:YF01234
        id:YF00301
        id:YF02016
    S14328
        id:YF04242
        id:YF00101
        id:YF01010
\end{verbatim}

Each line of the tree contains one or more SNPs or a sample
ID. Subclades and samples are indented by using tabs. Each
sample starts with \texttt{id:} followed by the ID. In our case
these are typical YFull IDs but Phyloage supports Family Tree
DNA data as well. Phyloage uses the Phylofriend
\cite{Phylofriend} program for data import and many
calculations as well. See the Phylofriend User Guide
\cite{PhylofriendUserGuide} for the details of the supported
input formats.

To create a results tree:

\begin{enumerate}
\item Save the input tree to a file, for example \emph{tree.txt}.
\item Download the Y-STR results for the samples from YFull
    and put them into a separate directory, for example
    \emph{allsamples}
\item Execute the following command from a command line
    (the file \emph{111-average.txt} can be found in the
    \emph{mutationrates} directory of the Phylofriend program
    \cite{Phylofriend}):\\
\texttt{phyloage -treein tree.txt -treeout results.txt\\
-personsin allsamples -mrin 111-average.txt -gentime 32}
\end{enumerate}

The results will be stored in a file named \emph{results.txt}.
We have used average mutation rates for 111 markers and a
generation time of 32 years.


\subsubsection*{Results tree}

Now the file \emph{results.txt} contains a tree with TMRCA
estimates that looks like this:

\begin{verbatim}
CTS4528, S1200, STRs Downstream: 145, formed: 4644, TMRCA: 4644
    S11481, STR-Count: 140, STRs Downstream: 51, formed: 6141, TMRCA: 1647
        id:YF01234, STR-Count: 30
        id:YF00301, STR-Count: 73
        id:YF02016, STR-Count: 52
    S14328, STR-Count: 17, STRs Downstream: 81, formed: 3148, TMRCA: 2597
        id:YF04242, STR-Count: 72
        id:YF00101, STR-Count: 91
        id:YF01010, STR-Count: 82
\end{verbatim}

Because we have used mutation rates for 111 markers, that
were calibrated by using generations, the \texttt{STR count}
is given in generations. It says for how long a Clade has
existed before it developed any subclades. The
\texttt{STRs Downstream} is a measure for the TMRCA.

\texttt{formed} denotes the age of the clade in years. It is
calculated by adding \texttt{STR count} to \texttt{STRs Downstream}
and multiplying the result by the generation time (the
\texttt{gentime} parameter) and an additional calibration
factor (1 by default).

\texttt{TMRCA} is the same as \texttt{STRs Downstream}. It is
just multiplied by the generation time and the calibration
factor.


\subsection{Mutation counting on the 500 marker scale}

In the previous example we have used the average mutation rates
for 111 markers. It is a good idea to start with those markers
because they are well known and have been used for years.

YFull on the other hand reports up to 500 STR markers. Most persons
get about 400 values out of their Big Y test. So we like to use
them, but keep in mind that we do not have a long time experience
with the 500 marker scale.

The input tree will be exactly the same as before but this time
we execute the command:

\vspace{1ex}\noindent
\texttt{phyloage -treein tree.txt -treeout results.txt\\
-personsin=allsample -mrin=500-count.txt -cal=39 }
\vspace{1ex}

\noindent The above command will do mutation counting using
all markers available and then upscale the results to 500
markers. The \texttt{500-count.txt} contains the mutation rates
for marker counting. It can be found in the Phylofriend
\cite{Phylofriend} \emph{mutationrates} directory.

Currently it seems like one mutation counts as 39 years
using Phylofriend's mutation model. So a calibration factor
of 39 is used.


\subsection{Adding Family Tree DNA results}

It is possible to add Family Tree DNA results to the input
tree. This is very useful if some persons only did test on
the 67 or 111 marker scale. It is also possible to add 12 or
37 marker results but the average mutation rates are different
and the margins of error are very high. So I do not recommend
it.

To add Family Tree DNA results, just insert their IDs
(kit numbers) into the input tree. In this example we add
12345 and 67890 to the input tree:

\begin{verbatim}
CTS4528, S1200
    S11481
        id:YF01234
        id:YF00301
        id:YF02016
        id:12345
        id:67890
    S14328
        id:YF04242
        id:YF00101
        id:YF01010
\end{verbatim}

To create the results tree:

\begin{enumerate}
\item Save the input tree to the file \emph{tree.txt}.
\item Put the YFull results into a directory named
	\emph{yfull}.
\item Save the Family Tree DNA results in a spreadsheet
	called \emph{cts4528.csv}. The first column of the 
	spreadsheet must contain the Family Tree DNA kit numbers.
	For additional details about the file format, please
	consult the Phylofriend User Guide \cite{PhylofriendUserGuide}.
\item Execute the following command from a command line:\\
	\texttt{phyloage -treein tree.txt -treeout results.txt\\
	-personsin yfull,cts4528.csv -mrin 111-average.txt\\
	-gentime 32}
\end{enumerate}

The results can be found in the file \emph{results.txt}.
Of course it is also possible to build a tree only from
Family Tree DNA samples and completely leave out YFull
results.


\subsection{Pure mutation counting}

If you do not have files containing detailed genetic results,
but you know the number of the individual (private) STR
mutations, it is possible to use the input tree for counting.
Just add the STR numbers to the tree like this (you can also
use this method for SNP counting):

\begin{verbatim}
CTS4528, S1200
    S11481, STR-Count: 2
        id:YF01234, STR-Count: 30
        id:YF00301, STR-Count: 40
        id:YF02016, STR-Count: 50
    S14328, STR-Count: 2
        id:YF04242, STR-Count: 30
        id:YF00101, STR-Count: 40
        id:YF01010, STR-Count: 50
\end{verbatim}


In this example we assume that one mutation counts as 100
years. Thus we use a calibration factor of 100. To create
the result tree, type:

\vspace{1ex}\noindent
\texttt{phyloage -treein tree.txt -treeout results.txt -cal 100}
\vspace{1ex}

\noindent
The result tree contains TMRCA estimates for all clades:

\begin{verbatim}
CTS4528, S1200, STRs Downstream: 42, formed: 4200, TMRCA: 4200
    S11481, STR-Count: 2, STRs Downstream: 40, formed: 4200, TMRCA: 4000
        id:YF01234, STR-Count: 30
        id:YF00301, STR-Count: 40
        id:YF02016, STR-Count: 50
    S14328, STR-Count: 2, STRs Downstream: 40, formed: 4200, TMRCA: 4000
        id:YF04242, STR-Count: 30
        id:YF00101, STR-Count: 40
        id:YF01010, STR-Count: 50
\end{verbatim}






