\section{Examples}

\subsection{Using YFull Results}

\subsubsection*{SNP Input Tree}

Before you can start you need to create a phylogenetic tree,
that contains SNPs and the IDs of the genetic samples. The
file format is text based and looks like this:

\begin{verbatim}
// This is an example tree.
// Comments begin with //.

CTS4528, S1200
    S11481
        id:YF01234
        id:YF00301
        id:YF02016
    S14328
        id:YF04242
        id:YF00101
        id:YF01010
\end{verbatim}

Each line of the tree contains one or more SNPs or a sample
ID. Subclades and samples are indented by using tabs. Each
sample starts with \emph{id:} followed by the ID. In our case
these are typical YFull IDs but Phyloage supports Family Tree
DNA data as well. Phyloage uses the Phylofriend
\cite{Phylofriend} program for data import and many
calculations as well. See the Phylofriend User Guide
\cite{PhylofriendUserGuide} for the details of the supported
input formats.

To create a results tree:

\begin{enumerate}
\item Save the input tree to a file, for example \emph{tree.txt}.
\item Download the Y-STR results for the samples from YFull
	and put them into a separate directory, for example
	\emph{allsamples}
\item Execute the following command from a command line:\\
\texttt{phyloage -treein tree.txt -treeout results.txt\\
-personsin allsamples -mrin 111-average.txt -gentime 32}
\end{enumerate}

The results will be stored in a file named \emph{results.txt}.
We have used average mutation rates for 111 markers and a
generation time of 32 years.


\subsubsection*{Results Tree}

Now the file \emph{results.txt} contains a tree with TMRCA
estimates that looks like this:

\begin{verbatim}
CTS4528, S1200, STRs Downstream: 145, formed: 4644, TMRCA: 4644
    S11481, STR-Count: 140, STRs Downstream: 51, formed: 6141, TMRCA: 1647
        id:YF01234, STR-Count: 30
        id:YF00301, STR-Count: 73
        id:YF02016, STR-Count: 52
    S14328, STR-Count: 17, STRs Downstream: 81, formed: 3148, TMRCA: 2597
        id:YF04242, STR-Count: 72
        id:YF00101, STR-Count: 91
        id:YF01010, STR-Count: 82
\end{verbatim}

Because we have used mutation rates for 111 markers, that
were calibrated by using generations, the \emph{STR count}
is given in generations. It says for how long a Clade has
existed before it developed any subclades. The
\emph{STRs Downstream} is a measure for the TMRCA.

\emph{formed} denotes the age of the clade in years. It is
calculated by adding \emph{STR count} to \emph{STRs Downstream}
and multiplying the result by the generation time (the
\emph{gentime} parameter) and an additional calibration
factor (1 by default).

\emph{TMRCA} is the same as \emph{STRs Downstream}. It is
just multiplied by the generation time and the calibration
factor.


% \subsection{Using Family Tree DNA Results}

% \subsection{Pure Mutation Counting}








