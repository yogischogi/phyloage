\section{First example}

\subsection{Input Tree}

Before you can start you need to create a phylogenetic tree,
that contains SNPs and the IDs of the genetic samples. The
file format is text based and looks like this:

\vspace{1em}
\noindent
// This is an example tree.\\
// Comments begin with //.
\vspace{1em}

\noindent L51\\
\hspace*{4ex}P310\\
\hspace*{8ex}P311\\
\hspace*{12ex}P312\\
\hspace*{16ex}id:YF00009\\
\hspace*{16ex}id:YF00301\\
\hspace*{16ex}id:YF02016\\
\hspace*{12ex}U106\\
\hspace*{16ex}id:YF00815\\
\hspace*{16ex}id:YF01234\\
\hspace*{16ex}id:YF05678\\
\hspace*{12ex}S1194\\
\hspace*{16ex}CTS4528\\
\hspace*{20ex}id:YF04242\\
\hspace*{20ex}id:YF00101\\
\hspace*{20ex}id:YF01010
\vspace{1em}

The tree is indented by using tabs. Each sample starts with
\emph{id:} followed the ID. In our case these are typical
YFull IDs but Phyloage supports Family Tree DNA data as well.
Phyloage uses the Phylofriend \cite{Phylofriend} program for
data import and many calculations as well. See the Phylofriend
User Guide \cite{PhylofriendUserGuide} for the details of
the supported input formats.

You can safe the input tree as \emph{tree.txt}. Next you need to
put the files containing the sample data into a directory for
example \emph{allsamples}. Now you can invoke the program by

\vspace{1em}
\noindent
\texttt{phyloage -treein tree.txt -treeout results.txt\\
-personsin allsamples -mrin 111-average.txt -gentime 32}
\vspace{1em}

\noindent The results will be stored in a file named \emph{results.txt}.
We have used average mutation rates for 111 markers and a
generation time of 32 years.












