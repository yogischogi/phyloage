\section{Introduction}

Phyloage is an experimental program for genetic genealogy
and history research. It combines SNP and Y-STR results to
calculate TMRCA (Time To Most Recent Common Ancestor)
estimates from STR mutational differences.

Since the emergence of next generation sequencing SNP based
phylogenetic trees are rapidly evolving and less attention
is paid to Y-STR mutations. But if we look at the properties
of SNP mutations in detail, we see that they are not without
drawbacks.

\begin{itemize}
\item SNP mutations are extremely stable. Phylogenetic trees
	based on SNPs are highly reliable.
\item Not all SNPs are useful for genealogical purposes. This
	is problematic for TMRCA calculations because wrong SNPs
	lead to false results.
\item For TMRCA calculations the margins of error are rather large.
	YFull has tried to include only genealgical relevant SNPs
	into their phylogenetic tree \cite{YFullTree} and estimates
	one mutation every 140 years \cite{YFullMutationRate}.
\end{itemize}

What we want is more precise estimates. This could be achieved
by using another kind of mutation, the well known Y-STRs. Their
properties are rather different from SNPs.

\begin{itemize}
\item STRs have been used for genetic genealogy for a long
	time. Some specific marker sets are well known and many
	people have tested.
\item STRs mutate back and forth. It is impossible to create
	reliable phylogenetic trees that go back deep into
	history. Furthermore the back and forth mutations lead
	to a saturation effect that invalidates time estimates
	for long time spans (thousands of years).
\item Next generation sequencing reveals results for up to
	500 markers. This should give us enhanced precision for
	TMRCA estimates.
\end{itemize}

So what does Phyloage do? The basic idea is to start with an
SNP based phylogenetic tree, insert the Y-STR results and
calculate modal haplotypes for each node of the tree. 

In theory this should reduce saturation effects and give
better time estimates for both deep history and genealogical
time frames.

If you use this program, please remember that it is still
highly experimental. At the time I am writing this, there
is no person on earth who has a long time experience with
500 Y-STR markers.

Have fun experimenting!

\vspace{1em} Dirk



