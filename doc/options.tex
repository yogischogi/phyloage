\section{Command line options}

Command line options may be given in arbitrary order.
Parameters may be specified by using a space or equals sign.
For example the following options are identical:
\texttt{-treein=mytree}, \texttt{-treein mytree}.

\begin{description}
\item[-help] Prints available program options.

\item[-treein] Filename of the SNP based phylogenetic tree.
\item[-treeout] Filename of the results tree in text format.
\item[-topdown] Specifies if the program should perform a top
    down recalculation of the age estimates on a tree. This 
    should yield better results. Default value is \texttt{-topdown=true}.
\item[-personsin] Filename or directory of files containing the
	persons' Y-STR values. If this is a single file it must contain
	results for multiple persons. The input file format is CSV
    (comma separated values) or text format.

	If a directory is provided for input it must contain multiple
	files in YFull format, each file containing the results for
	a single person. The person's ID is extracted from the filename.

	\texttt{personsin} supports multiple file names separated by
	commas.
\item[-mrin] Filename of the mutation rates to use.
\item[-model] Mutation model to use. This may be \texttt{hybrid}
	or \texttt{infinite}. \texttt{hybrid} uses uses stepwise counting
	for most markers except for the palindromic ones. 
	\texttt{infinite} uses the infinite alleles mutation model for
	all markers.
\item[-gentime] Generation time.
\item[-cal] Calibration factor.
\item[-offset] An offset that is added to all calculated ages.
\item[-subclade] Selects a branch of the tree specified by an SNP.
\item[-htmlout] Output filename for Y-STR markers in HTML format.
\item[-statistics] Prints out marker statistics.
\item[-inspect] Prints out details about the specified SNPs or
	sample IDs. The search terms must be specified by a comma
	separated list, for example \texttt{-inspect=CTS4528,S11481,S14328}.
\item[-trace] Prints out a phylogenetic tree that contains the
	mutational values for the specified Y-STR markers. Example:
	\texttt{-trace=DYS393,DYS19}.
\item[-method] Method to be used for calculating modal haplotypes:
	\texttt{phylofriend} or \texttt{parsimony}. The default method
	is \texttt{parsimony}, which uses a maximum parsimony algorithm.
\item[-stage] Processing stage for the parsimony algorithm. This
	should be used for debugging or to see in detail what the algorithm
	does. The following stages are valid:
	\begin{description}
	\item[1] Calculates haplotype values that strictly satisfy
		the maximum parsimony criterion.
	\item[2] Calculates average haplotype values for the rest
		by using real numbers.
	\item[3] Replaces real numbers by real world mutation values
		and sets all uncertain values to -1.
		This stage is only for visualization and debugging.
	\item[4] Forces all values to real world mutation values.
	\end{description}
\end{description}

